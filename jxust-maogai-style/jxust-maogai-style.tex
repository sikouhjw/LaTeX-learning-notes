\usepackage[margin=2cm,left=2.5cm]{geometry}%设置页面布局

\setmainfont{Times New Roman}%设置英文衬线字体为 Times New Roman,文章中主要使用的字体

% \setCJKmainfont{Source Han Serif SC}%设置中文衬线字体,文章中主要使用的字体
%粗体和斜体都是对英文字体而言的,中文字体一般没有对应的粗体和斜体。
%在 Windows 系统下,xetex 引擎默认调用中易字库 + 微软雅黑字体(见 ctex 宏包),
%默认在正文中使用中易宋体,即 word 中的宋体,
%该字体没有对应的粗体和斜体,只能通过使用伪粗来达到加粗的效果,word 的加粗就是这样。
%所以这里不使用中易宋体,改用思源宋体。思源宋体有不同的粗细,可以来表示对应的粗体,
%所以这里使用思源宋体。思源宋体需要自行安装。下载链接
% https://github.com/adobe-fonts/source-han-serif
%右键安装或者复制到字体文件夹中。

%以下代码配置:使用中易宋体为默认字体,中易黑体为粗体,楷体为斜体。楷体可能是叫中易楷体。
\setCJKmainfont[ItalicFont={KaiTi},BoldFont={SimHei}]{SimSun}
%用 \bfseries 使用中易黑体,\itshape 使用楷体。

%以下代码配置:使用伪粗来表示对应的粗体。
\xeCJKsetup{EmboldenFactor={1}}%设置伪粗体的默认粗细程度,不加粗为 0,默认为 4。
\setCJKfamilyfont{Kai}[AutoFakeBold]{KaiTi}
\newcommand*{\kai}{\CJKfamily{Kai}}%之后可以使用 \bfseries\kai 使用伪粗楷体。
\setCJKfamilyfont{Song}[AutoFakeBold]{SimSun}
\newcommand*{\song}{\CJKfamily{Song}}%之后可以使用 \bfseries\song 使用伪粗中易宋体。

\usepackage[titles]{tocloft}%用于设置目录格式的宏包

%设置目录
%重定义目录标题格式,使目录以三号粗体居中,中间空两个小 4 号字体的间距。
\renewcommand{\contentsname}{\hfill\zihao{3} {\bfseries 目\hspace{24bp}录}\hspace*{\fill}}
%设置目录引导符,(我叫它引导符)
\renewcommand{\cftdot}{$\cdot$}%设置引导符为 $\cdot$
\renewcommand{\cftdotsep}{1}%设置引导符间距
\renewcommand{\cftsecleader}{\cftdotfill{1}}%为目录中的 section 添加引导符
%设置目录字体和缩进
\renewcommand{\cftsecfont}{\bfseries}%设置目录中的 section 字体为粗体。
\cftsetindents{section}{-24bp}{12bp}%修正 \ctexset 中设置标题产生的缩进
\cftsetindents{subsection}{0bp}{12bp}%修正 \ctexset 中设置标题产生的缩进
\cftsetindents{subsubsection}{0bp}{12bp}%修正 \ctexset 中设置标题产生的缩进


\setlength{\cftbeforesecskip}{0pt}%目录中 section 之前默认有额外的间距,让这个间距的值为 0。

%设置正文中的标题格式,具体效果见 ctex 宏包,不想写了。。。
%毕竟 ctex 宏包应该都看过了。
\ctexset{
    section={
        number={\chinese{section}},
        format={\raggedright\bfseries\zihao{4}},
        name={\hspace{24bp},},
        % aftername={},
        afterskip=0pt,
        beforeskip=6bp,
    },
    subsection={
        number={\arabic{section}.\arabic{subsection}},
        name={\hspace{24bp}},
        format={\raggedright\bfseries\zihao{4}},
        beforeskip=0pt,
        afterskip=0pt,
    },
    subsubsection={
        number={\thesubsection.\arabic{subsubsection}},
        name={\hspace{24bp}},
        format={\raggedright\bfseries\zihao{-4}},
        beforeskip=0pt,
        afterskip=0pt
    }
}





\usepackage{graphicx}%插图宏包
\graphicspath{{figures/}}%设置图片路径。

%定义封面样式
\renewcommand{\maketitle}{\begin{titlepage}
    \vspace*{\fill}%和后面的 \vspace*{\fill} 一起,使内容垂直居中。
    \centering%使内容水平居中。
    \includegraphics[width=0.7\paperwidth]{logo.jpg}\\%插入图片
    \vspace{12bp}\kai %使用楷体
    {\zihao{3}\bfseries 《毛泽东思想和中国特色社会主义理论体系概论》}\\ %课程名:毛概。
    \vspace{24bp}
    {\bfseries\zihao{2} 实\\[12bp] 践\\[12bp] 报\\[12bp] 告\\[12bp] }
    \vspace{24bp}
    \hspace*{72bp}%让 minipage 的内容靠右一点,建议看自己需要调整过。
    \begin{minipage}{12cm}
        \zihao{4}
        时间:\today\\
        实践题目:江西理工大学校园网络文化建设调查与分析\\
        组长:\hspace{2bp} \mbox{。。。 \quad 。。。}\\
        组员:\hspace{2bp} \parbox[t]{10cm}{。。。\quad 其他内容 。。。 \\
                。。。\quad 其他内容 。。。\\
                。。。\quad 其他内容 。。。\\
                。。。\quad 其他内容 。。。 }\\[12bp]
        专业班级:。。。。。。。。。\\
        评分:
    \end{minipage}
    \vspace*{\fill}%和前面的 \vspace*{\fill} 一起,使内容垂直居中。
\end{titlepage}}



\linespread{1.5}%设置行距为 1.5 倍。

\pagestyle{plain}%设置页眉页脚格式,无页眉,页码居中放在页脚。
