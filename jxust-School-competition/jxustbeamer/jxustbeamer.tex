% !TeX program = xelatex
% !TeX encoding = UTF-8

\documentclass{jxustbeamer}
\usetheme{PaloAlto}%这些是我喜欢的主题:Berkeley,CambridgeUS,Copenhagen,Hannover,Madrid,PaloAlto
\title{JXUST答辩模板示例}
\author{死抠\thanks{\href{https://github.com/sikouhjw/LaTeX-learning-notes}{\fbox{github链接}}}}
\institute{\LaTeX{}学院}
\date{\scriptsize {\today}}
%\titlegraphic{\includegraphics[width=0.20\textwidth]{jxust.png}%在第一页产生大的校徽
\logo{\includegraphics[width=1.3cm,height=1.3cm]{jxust.png}}%控制每页都产生校徽

\begin{document}
	\begin{frame}
		\titlepage
	\end{frame}

    \begin{frame}{目录}
    	 \tableofcontents[hideallsubsections]
    \end{frame}

\section{模板基本介绍}
    \begin{frame}{模板基本介绍}
        本模板是为江西理工大学数学建模竞赛答辩准备的,为了让我自己方便答辩,我编写了这个模板。
        
        要使用本模板的同学应该要阅读过 lshort-zh\cite{lshort-zh} 或者 \LaTeX{}入门\cite{刘海洋}的大部分内容, 会使用\LaTeX{}的基本命令,能够用\LaTeX{}编写自己的文档。
        
        因为我没答辩过,不知道示例要有什么内容,所以我只是把功能罗列了一遍。
    \end{frame}
\section{基本使用}
\subsection{更换主题}
\begin{frame}{更换主题}
	本模板在导言区注释了一部分的主题,读者可根据自身需要选择,也可用\LaTeX{}入门\cite{刘海洋}里的主题或者beamer宏包手册里的主题。
\end{frame}
\subsection{抄录命令的使用}
    \begin{frame}[fragile]{抄录命令的使用}
    %注意此处verbatim内的内容最左边不能有空格
\begin{verbatim}
若在一帧中有抄录环境,务必加上[fragile]
示例:
\ begin{frame}[fragile]{这是一个例子}
这是一个例子,\与begin之间不应该有空格,但是没空格会报错
\ end{frame}
\end{verbatim}
    \end{frame}
\subsection{数学公式}
\begin{frame}[fragile]{数学公式}

计算 $\displaystyle\lim_{(m,n)\to(+\infty,+\infty)}\sum_{j=1}^{m}\sum_{i=1}^{n}\frac{(-1)^{i+j}}{i+j}$

\end{frame}
\subsection{重新定义}
\begin{frame}[fragile]{重新定义}
	\begin{figure}[h]
		\begin{tabular}{|c|c|}
			\hline
			原命令 & 重新定义后\\
			\hline
			\verb|\,\mathrm{d}| & \verb|\dd|\\
			\hline
			\verb|\mathrm{e}| & \verb|\ee|\\
			\hline
			\verb|\mathrm{i}| & \verb|\ii|\\
			\hline
		\end{tabular}
	\end{figure}
\end{frame}
\subsection{图片的插入}
\begin{frame}{图片的插入}
	图片请放在figures文件夹中
	\begin{figure}[h]
		\centering
		\includegraphics[width=0.20\textwidth]{jxust.png}
	\end{figure}
	
\end{frame}
\subsection{列表的使用}
\begin{frame}{列表的使用}
	\begin{enumerate}
		\item 这是1
		\item 这是2
		\begin{enumerate}
			\item 这是嵌套了一次
		\end{enumerate}
	\end{enumerate}
\end{frame}

\subsection{彩色}
\begin{frame}{彩色}
	\textcolor{red}{这是红色}
	
	\textcolor{blue}{这是蓝色}
	
	\colorbox{yellow}{黄色盒子}
	
	\fcolorbox{black}{green}{黑框绿盒子}
\end{frame}

\subsection{盒子}
\begin{frame}{盒子}
	\begin{block}{蓝色}
		我是盒子
	\end{block}

    \begin{alertblock}{红色}
    	我也是盒子
    \end{alertblock}

    \begin{exampleblock}{绿色}
    	巧了你们俩也是盒子
    \end{exampleblock}
\end{frame}

\subsection{代码环境}
\begin{frame}[fragile]{代码环境}
	matlab代码环境
\begin{lstlisting}[language=matlab]
rand('state',sum(clock)); % 初始化随机数发生器
p0=0;
tic % 计时开始
for i=1:10^6
x=randi([0,99],1,5); % 产生一行五列的区间[0,99]上的随机整数
[f,g]=mengte(x);
if all(g<=0)
if p0<f
x0=x; p0=f; % 记录下当前较好的解
end
end
end
x0,p0
toc % 计时结束
\end{lstlisting}
\end{frame}
\section{参考文献}
\begin{frame}{参考文献}
	\begin{thebibliography}{99}
		\bibitem{lshort-zh} China\TeX{} 论坛, 《 一份不太简短的 \LaTeXe{} 介绍》 (lshort-zh).
		\bibitem{刘海洋} 刘海洋. LaTeX入门. 北京:电子工业出版社, 2013.
	\end{thebibliography}
\end{frame}
\section{致谢}
\begin{frame}{致谢}
    \begin{block}{队友}
    	\centering
    	感谢我的队友lrc和jzw
    \end{block}
    \pause
    \begin{block}{老师}
    	\centering
    	感谢我的指导老师yhb
    \end{block}
    \pause
    \begin{block}{\LaTeX{}指导}
    	\centering
    	感谢xaj在\LaTeX{}上的指导
    \end{block}
    \pause
	\begin{center}
		志存高远\hspace{1pc}责任为先
	\end{center}
    \pause
    \begin{figure}[h]
    	\centering
    	\includegraphics[width=0.15\textwidth]{jxust.png}
    \end{figure}
\end{frame}
\end{document}